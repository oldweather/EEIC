% CORRAL writeup

%\documentclass[a4paper,11pt, twocolumn]{article}
\documentclass[CP]{copernicus}
\usepackage[pdftex]{graphicx}
%\usepackage{lineno}

\begin{document}
\title{Arctic marine climate of the early nineteenth century}

\runningauthor{BROHAN ET AL}
% Author names in capital letters,
\runningtitle{ARCTIC WEATHER 1810--25 - DRAFT 5}
% Shorter version of title entered in capital letters

\author[1]{Philip Brohan}
\author[2]{Catharine Ward}
\author[1]{Gail Willetts}
\author[3]{Clive Wilkinson}
\author[1]{Rob Allan}
\author[2]{Dennis Wheeler}
\date{Draft 5 --- 1 February 2010}

\affil[1]{Met Office Hadley Centre, Exeter, UK.}
\affil[2]{Sunderland University, Sunderland, UK}
\affil[3]{University of East Anglia, Norwich, UK}

\correspondence{Philip Brohan\\ (philip.brohan@metoffice.gov.uk)}
\firstpage{1}

\maketitle

\begin{abstract}

The climate of the early nineteenth century is likely to have been significantly cooler than that of today, as it was a period of low solar activity (the Dalton minimum) and followed a series of large volcanic eruptions. Proxy reconstructions of the temperature of the period do not agree well on the size of the temperature change, so other observational records from the period are particularly valuable. Weather observations have been extracted from the reports of the noted whaling captain William Scoresby Jnr., and from the records of a series of Royal Navy expeditions to the Arctic, preserved in the UK National Archives. They demonstrate that marine climate in 1810--25 was marked by consistently cold summers, with abundant sea-ice. But although the period was significantly colder than the modern average, there was a lot of variability: In the Greenland Sea the summers following the Tambora eruption (1816 and 1817) were noticeably warmer, and had lower sea-ice coverage, than the years immediately preceding them; and the sea-ice coverage in Lancaster Sound in 1819 and 1820 was low even by modern standards.
\end{abstract}

\introduction

Instrumental records that adequately reflect large-scale climate change go back only to about 1850 (\citet{brohan06hadcrut3, allan06meansealevelpressure}) when the first national meteorological services were founded and systematic data collection began. There are limited instrumental records going further back \citep{jones03seas,parker92CET}, but for earlier periods we are mostly reliant on reconstructions from proxy observations (e.g. \citet{jones09millennium}) and non-instrumental observations (e.g. \citet{wheeler06cliwoc}). These reconstructions have large uncertainties, and the uncertainties are particularly large for the climate of the early 19th century \citep{jansen07ipcc}: when a combination of low solar activity (the Dalton minimum) and a series of large volcanic eruptions (notably the Tambora eruption of 1815) produced a cool period - but it is not known either how much the global mean temperature fell or how the climate changes varied around the world.

Quantifying the climate change of the early nineteenth century would both be of value for climate model development and validation \citep{wagner05dalton} and would help constrain proxy reconstructions of longer-term climate change. Such a quantification is possible because, not only are there instrumental observations for the period \citep{chenoweth01,harrington_1992}, but there is also potential to obtain new and additional observations from hitherto overlooked archive sources \citep{brohan09digitisation}. Although Arctic voyages in the early 19th century were not chiefly concerned with scientific investigation, many experiments were made and much valuable geo-magnetic, meteorological and oceanographic data was collected. 

A recent project ({\tt http://www.corral.org.uk}) has catalogued and photographed the National Archive's collection of Royal Navy logbooks and journals of ships on voyages of exploration and discovery (catalogue section ADM55). Some of these logbooks contain instrumental weather observations, including those from a group of Arctic expeditions in the early nineteenth century, and some of these observations have been digitised as part of the project. Meanwhile, the international ACRE initiative ({\tt http://www.met-acre.org}) has augmented this data by digitising additional weather records from the published accounts of the same expeditions, and from the published reports of the noted whaling captain William Scoresby Junior \citep{jackson_2003}. These records, so far scarcely examined by climatologists, provide a unique insight into the climate of the Arctic of this period, long before regular observations were established --- and it was the remarkable William Scoresby, a whaler with a scientific education, who first made systematic observations of the temperature of the region.

\section{William Scoresby}

William Scoresby Junior followed the example of his father (also William Scoresby) in becoming a successful and distinguished whaling captain; but he improved on his father's example by recording temperatures and weather encountered on his voyages, and publishing accounts of them. In each summer between 1810 and 1818 Scoresby was to be found, hunting whales, close to the edge of the ice in Fram Strait; and he published an account of positions, air temperature and pressure measurements in \citet{scoresby_1820}. In 1822 he extended his voyage to the East coast of Greenland, and again published an account in \citet{scoresby_1823}. The routes of these voyages are shown in figure \ref{FS1}.
\begin{figure*}[!hbp]
\begin{center}
\includegraphics[angle=0, width=0.8\textwidth]{../figures/Routes_scoresby}
\caption{Routes of William Scoresby's whaling voyages. Except for the trip to Greenland in 1822, all the voyages were similar in the area covered, so only 5 of the 10 annual voyages are plotted, to avoid muddying the figure. The broad grey line is the September sea-ice edge from \citet{rayner03HadISST1}, median value over 1979--2004.}
\label{FS1}
\end{center}
\end{figure*}

Whales were easiest to find at the edge of the sea-ice, so the whaling fleet mostly operated as far north as it could get. Figure \ref{FS2} compares Scoresby's measured temperatures with the range of values in a modern (1979--2004) air temperature climatology, and also shows the range of sea-ice expected at the position of the ship, from a modern sea-ice climatology for the same period (see section \ref{mcomp}). As the manner in which the temperature measurements was made is not known, it is possible that some differences between them and the climatology are an artifact of observational practice (section \ref{atb}), so a direct comparison is not appropriate. But, as the measurements are consistently from the same area, they can be used to investigate year-to-year changes.

In a cold year, the measured temperatures will move down with respect to the climatology, and the ship is likely to be forced south by the increasing sea-ice, so the climatological sea-ice along the route of the ship will be reduced. In a warm year, the opposite will occur.
\begin{figure*}[!hbp]
\begin{center}
\includegraphics[angle=0, width=0.8\textwidth]{../../../../expeditions/voyages/scoresby_1807-18/figures/AT+I.pdf}
\caption{Weather observations from William Scoresby's whaling voyages, air temperatures from each of 10 seasons compared with modern values. The red points are the observations, the black and grey lines are the mean, max and min values from modern datasets (monthly averages interpolated to daily, 1979-2004; AT from \citet{rigor97poles}, sea-ice from \citet{rayner03HadISST1}).}
\label{FS2}
\end{center}
\end{figure*}
A remarkable feature of Scoresby's observations is that the summer of 1816 (the `year without a summer' following the Tambora eruption) was unusually {\it warm} in the Greenland sea; as was the following summer (figure \ref{FS2}). Writing to Sir Joseph Banks, president of the Royal Society, William Scoresby stated: `I observed on my last voyage (1817) about 2000 square leagues, (18,000 square miles) of the surface of the Greenland seas, included between the parallels of 74$^\circ$ and 80$^\circ$, perfectly void of ice, all of which disappeared within the last two years.' \citep{barrow_1846}. This combination of a clear statement from an experienced observer, and instrumental observations of air temperature which indicate the same changes, makes the Scoresby papers a compelling and precise source of information on climate variability.

\section{Naval Expeditions}

In addition to the interest in the region from the whaling and economic point-of-view, The British Government, and institutions such as the Royal Society, were keen to sponsor more ambitious scientifically and politically driven Arctic voyages. The hazards of such voyages were not to be taken lightly, but considerable rewards were offered: an Act of Parliament provided a reward of \pounds 5,000 to anyone able to penetrate beyond 89$^\circ$ North, and \pounds 20,000 to any mariner who could discover a route from the Atlantic to the Pacific in the Northern Hemisphere --- the fabled `North-West passage'. Unfortunately for the government, this offer was not taken up seriously despite the fabulous sums on offer: The Greenland whaling ships were not insured to operate outside of the normal fishing grounds, and the ships' masters had to take an oath on each voyage that they would not profit from any activity other than fishing \citep{barringdon_1818}. So the British Government found themselves having to call on the Royal Navy.

At the turn of the nineteenth century, the most recent official expedition was that led by Constantine Phipps in 1773 \citep{phipps_1775}: Daines Barrington, a prominent member of the Royal Society, had proposed the voyage, and the Admiralty provided and equipped HMS {\it Racehorse} and HMS {\it Carcass} - with orders to approach the Pole along the longitude of Greenwich. The two ships had managed to reach only 80.6$^\circ$ North in longitude 2.03$^\circ$ East. Fresh activity was spurred in 1818 by the reports of changed ice conditions both in the Arctic and at lower latitudes: as well as Scoresby's findings, ships sailing to Halifax and Newfoundland in 1815--17 were reporting sightings of `ice islands' as far south as the 40th parallel. At the same period, whaling ships in the Davis strait reported an unusual number of icebergs and floes drifting southward \citep{barringdon_1818}. The possibility of a clear northern passage to the Pacific could not be ignored, and two new naval expeditions set out in 1818; to be followed by several others in the years following.

Weather observations have been digitised from five of these expeditions (table \ref{T1}) all dating from between 1818 and 1825, and covering the Far North Atlantic and Canadian Arctic (figure \ref{F1}). Some of these logbook records have been previously recognised as a valuable source of climate data \citep{ward06instruments}: The air temperature measurements made while the ships were overwintering at fixed locations have been extracted and analysed by \citet{przybylak05Arctic} and \citet{przybylak09Arctic}. But the ships also measured air pressure, sea temperature, wind speed and direction, and sometimes humidity; and observations were made while the ships were at sea as well as when in winter harbour. 

\begin{table*}[!hbp]
\begin{center}
%\begin{tabular}{|c|p{3cm}|p{2cm}|}
\begin{tabular}{|c|p{3cm}|p{3cm}|p{6cm}|}
\hline
{\bf Date} & {\bf Expedition Leader} & {\bf Ships} & {\bf Expedition aim}\\
\hline
1818 & John Ross & {\it Isabella} and {\it Alexander} & Pacific via Baffin Bay and North-West Passage \\
\hline
1818 & David Buchan & {\it Dorothea} and {\it Trent} & Pacific via Fram Strait and North Pole \\
\hline
1819-20 & William Parry & {\it Hecla} and {\it Griper} & Northwest Passage via Lancaster sound \\
\hline
1821-3 & William Parry & {\it Fury} and {\it Hecla} & Northwest Passage via Hudson's Bay \\
\hline
1824-5 & William Parry & {\it Fury} and {\it Hecla} & Northwest Passage via Lancaster sound \\
\hline
\end{tabular}
\vspace{0.5cm}
\caption{Expeditions from which observations were digitised}
\label{T1}
\end{center}\end{table*}

\begin{figure*}[!hbp]
\begin{center}
\includegraphics[angle=0, width=0.8\textwidth]{../figures/Routes}
\caption{Routes of the Royal Navy expeditions providing new observations. Filled circles mark overwintering sites.}
\label{F1}
\end{center}
\end{figure*}

\subsection{HMS Dorothea, 1818}

According to the instructions issued to Captain David Buchan of HMS {\it Dorothea} in 1818, 'the passage between the Atlantic and the Pacific is the main object of your mission.' Buchan's instructions were quite specific. He was to sail between Greenland and Spitzbergen --- the seas around the latter having been generally free of ice as far as 84$^\circ$N ... 'there is reason to expect that the sea may continue still more open northward, ... In this event you will steer due north ... to reach the North Pole. ... on leaving the Pole, you will endeavour to shape a course direct to Behring's Strait.' \citep{buchan_1818}. In the event of obstruction by land or ice, Buchan was directed to pass between Greenland and the east coast of America, into Baffin's Bay, and then to the Davis Strait and home. He was accompanied by HMS {\it Trent}, commanded by John Franklin. In the event they failed to penetrate the pack ice in Fram Strait, spending only one summer in the Arctic.

Records of the expedition include a logbook containing the Dorothea's positions (\citet{dorothea_deck_log_1818}, see figure \ref{F1}), and another containing observations of pressure, air temperature, humidity, wind speed and direction \citep{dorothea_met_log_1818} (figure \ref{F4}).
\begin{figure*}[!hbp]
\begin{center}
\includegraphics[angle=0, width=0.8\textwidth]{../../../ships/dorothea_1818/figures/All}
\caption{Weather observations from HMS {\it Dorothea}, compared with modern values. The red points are the expedition's observations, the black and grey lines are the mean, max and min values from modern datasets (monthly averages interpolated to daily, 1979-2004; sea-ice from \citet{rayner03HadISST1}, AT from \citet{rigor97poles}, and pressure from \citet{allan06meansealevelpressure}).}
\label{F4}
\end{center}
\end{figure*}

\subsection{HMS Isabella, 1818}

A simultaneous, sister expedition was led by John Ross, in HMS {\it Isabella}; in company with HMS {\it Alexander}, commanded by William Parry. Ross had the same aim --- Buchan's instructions stated 'you are to fix with Captain Ross, to whom the other expedition is entrusted, upon a rendezvous in the Pacific' \citep{buchan_1818} --- but Ross was instructed to seek a Northwest Passage via Baffin Bay. In the event they turned back at the entrance to Lancaster Sound, spending only one summer in the Arctic. The route of the expedition is documented in the Isabella's deck log (\citet{isabella_deck_log_1818}, see figure \ref{F1}), and the ship's meteorological logbook \citep{isabella_met_log_1818} contains instrumental observations of sea temperature, air temperature and pressure, supplemented by observations of wind speed and direction (figure \ref{F3}).
\begin{figure*}[!hbp]
\begin{center}
\includegraphics[angle=0, width=0.8\textwidth]{../../../ships/isabella_1818/figures/All}
\caption{Weather observations from HMS {\it Isabella}, compared with modern values. The red points are the expedition's observations, the black and grey lines are the mean, max and min values from modern datasets (monthly averages interpolated to daily, 1979-2004; SST and sea-ice from \citet{rayner03HadISST1}, AT from \citet{rigor97poles}, and pressure from \citet{allan06meansealevelpressure}).}
\label{F3}
\end{center}
\end{figure*}

\subsection{HMS Hecla, 1819--20}

William Parry was not impressed by Ross' decision not to venture down Lancaster Sound, and in 1819 he was appointed to the command of a follow-up expedition. Parry commanded HMS {\it Hecla}, and was accompanied by HMS {\it Griper}, commanded by Matthew Liddon. They travelled through Baffin Bay and Lancaster Sound, reaching Melville Island, where they spent the winter before turning back. Several logs from this expedition are in the ADM55 collection; but the meteorological logbooks are not among them, so meteorological observations have been taken from a published account of the expedition instead \citep{parry_1819}. The meteorological abstracts in Parry's account give the ship positions (figure \ref{F1}), and daily maximum, minimum and mean air temperatures, sea temperatures and air pressures (figure \ref{F5}). 
\begin{figure*}[!hbp]
\begin{center}
\includegraphics[angle=0, width=0.8\textwidth]{../../../ships/hecla_1819-20/figures/All}
\caption{Weather observations from HMS {\it Hecla} in 1819--20, compared with modern values. The red points are the expedition's observations, the black and grey lines are the mean, max and min values from modern datasets (monthly averages interpolated to daily, 1979-2004; SST and sea-ice from \citet{rayner03HadISST1}, AT from \citet{rigor97poles}, and pressure from \citet{allan06meansealevelpressure}). The grey band marks the period the expedition spent in winter harbour.}
\label{F5}
\end{center}
\end{figure*}

\subsection{HMS Hecla, 1821--3}

In 1821, Parry led a second second Northwest Passage expedition: commanding HMS {\it Fury}, and accompanied by HMS {\it Hecla} commanded by George Francis Lyon. They spent two winters in Foxe Basin, returning in the Autumn of 1823 after failing to penetrate the permanent ice in the strait between Foxe Basin and the Gulf of Boothia (named Fury and Hecla strait) (\cite{parry_1821} and figure \ref{F1}). The ADM55 collection includes two meteorological logbooks for the {\it Hecla} \citep{hecla_met_log_1821,hecla_met_log_1822-3} providing air and sea temperatures, air pressures, and wind speed and direction (figure \ref{F2}). The ship's position information (latitude and longitude) has been taken from from \citet{hecla_deck_log_1821-3}. 
\begin{figure*}[!hbp]
\begin{center}
\includegraphics[angle=0, width=0.8\textwidth]{../../../ships/hecla_1821-3/figures/All}
\caption{Weather observations from HMS {\it Hecla} in 1821--3, compared with modern values. The red points are the expedition's observations, the black and grey lines are the mean, max and min values from modern datasets (monthly averages interpolated to daily, 1979-2004; SST and sea-ice from \citet{rayner03HadISST1}, AT from \citet{rigor97poles}, and pressure from \citet{allan06meansealevelpressure}). The grey bands mark the periods the expedition spent in winter harbours.}
\label{F2}
\end{center}
\end{figure*}

\subsection{HMS Fury and Hecla, 1824--5}

HMS {\it Fury} and HMS {\it Hecla} also formed William Parry's third Northwest Passage expedition: Parry commanded the {\it Hecla}, and Henry Parkyns Hoppner the {\it Fury}. They followed a similar route to the 1819--20 expedition, but got only as far as Port Bowen before having to stop for the winter \citep{parry_1824}. The {\it Fury} was damaged by ice and had to be abandoned, forcing the expedition to return in the summer of 1825. Several logbooks from this expedition are in the ADM55 collection, but only \citet{fury_met_log_1824} contained any meteorological observations, and these only covered the first part of the voyage. So again observations have been taken from a published account of the expedition \citep{parry_1824} (figure \ref{F6}).

\begin{figure*}[!hbp]
\begin{center}
\includegraphics[angle=0, width=0.8\textwidth]{../../../ships/hecla_1824-5/figures/All}
\caption{Weather observations from HMS {\it Hecla} and {\it Fury} in 1824--5, compared with modern values. The red and blue points are the expedition's observations (red from \citet{parry_1824}, and blue from \citet{fury_met_log_1824}), the black and grey lines are the mean, max and min values from modern datasets (monthly averages interpolated to daily, 1979-2004; SST and sea-ice from \citet{rayner03HadISST1}, AT from \citet{rigor97poles}, and pressure from \citet{allan06meansealevelpressure}). The grey band marks the period the expedition spent in winter harbour.}
\label{F6}
\end{center}
\end{figure*}

\section{Comparing the expedition records with modern climatologies}
\label{mcomp}

The Arctic is notably sensitive to changes in global climate, and one of the main interests in the observations is in comparisons with more recent climate. As the ships move about, it is impractical to compare their measurements to modern fixed station records --- instead gridded climatologies have been used:
\begin{itemize}
\item Arctic air temperatures for 1979--2004 are provided by the POLES dataset \citep{rigor97poles}.
\item Sea-temperature and sea-ice coverage for 1870--present are provided by the HadISST dataset \citep{rayner03HadISST1}.
\item Sea-level pressures for 1850--2004 are provided by the HadSLP2 dataset \citep{allan06meansealevelpressure}.
\end{itemize}
For each observation, the gridded climatological value on the day of the measurement, at the location of the ship, has been extracted for each year from 1979--2004. The mean, maximum and minimum of the yearly values is shown in figures \ref{F3}--\ref{F6} along with the observed values. This indicates the range of observations to be expected from a ship reproducing the route of each expedition at the end of the twentieth century. As the gridded climatologies used are all monthly (interpolated to daily), the short-term variability (diurnal and day-to-day changes) is not included --- the range shown is of seasonal and interannual variability.

\section{Known biases}

The climatological comparisons show the observations expected from observations made with modern instruments and current observing practices. The expeditions were furnished with the best instruments of the time, and measurement errors are expected to be small \citep{ward06instruments}, but changes in observing practices will produce some changes.

\subsection{Sea temperature}

Exactly how the sea-surface temperature measurements were made is not known in most cases, but they are likely to be measurements of samples taken in some form of bucket \citep{prestwich74subsurface}. The SST bias adjustments of \citet{folland95buckets} suggest that bucket measurement biases in the Arctic are likely to be small ($<0.1^\circ C$).

\subsection{Air pressure}

The barometer readings have been corrected for latitude (gravity), but probably not for temperature, or for height of barometer above sea-level --- neither the attached thermometer temperature, nor the barometer location on the ships are exactly known. It is likely that the barometers were kept in the captain's cabin, and so protected from the most extreme cold. A temperature variation of 20$^\circ$C would give a barometric change of about 3.5hPa, so the observations may have seasonally varying biases of about this amount. Previous work on marine pressure observations for this period has found that readings are sometimes too low by around 5hPa --- the cause of this bias is still unknown \citep{allan06meansealevelpressure}. The two 1818 expeditions in this dataset demonstrate this bias (figures \ref{F3} and \ref{F4}), although the later expeditions do not.

\subsection{Air temperature}
\label{atb}

The air temperature measurements are subject to a number of biases:
\begin{itemize}
\item Thermometer exposure: The modern Stevenson-type screen did not become common until well into the twentieth century. The measurements may be biased by inadequate shielding of the thermometer.
\item Solar heating of the ship: The ship is likely to heat up in direct sunlight more than it's surroundings.
\item Direct heating from the ship: In the very cold conditions of the Arctic winter, the interiors of the ships were heated by coal fires. This was observed to bias the thermometers on deck. 
\end{itemize}
The officers and scientists making the observations were aware of these problems and took care to minimise and understand them. Parry noted that during the winter of 1819--20, the ship's deck thermometers (which provide the records analysed here) were around 2 to 5$^\circ F$ (1 to 2$^\circ C$) higher than measurements on shore. He attributed this difference to direct heating --- though subsequent expeditions did not reproduce this effect \citep{ward06instruments}. \citet{chenoweth00homogenization} investigated air temperature measurement biases in early 19th century marine observations and found ship-specific constant biases between -1.76$^\circ$C and 1.74$^\circ$C, and solar heating biases less than 1$^\circ$C even in the tropics (Arctic effects will be even less). The quality of the instruments used, and the care in taking the observations, is likely to be higher for the observations analysed here than those used in \citet{chenoweth00homogenization}, at least for the naval expeditions \citep{ward06instruments}, the observations are therefore unlikely to be seriously in error.

\subsection{Sea-ice}

The meteorological logbooks do not contain systematic, quantitative statements about sea-ice; but it is possible to infer the coverage from the movements of the ships. Because the ships were trying to make their way through ice-blocked seas they were operating for much of the time right at the edge of navigable ice, in particular the ships were surrounded by solid ice throughout their periods in winter harbour --- they generally had to resort to saws or explosives to break up the ice to get into and out of winter harbour locations \citep{parry_1819,parry_1821,parry_1824}. Sea-surface temperatures also give information on sea-ice: SSTs of less than about $2^\circ C$ are generally indicative of ice-cover, and by the time the SST reaches $-1.8^\circ C$ (the freezing point of seawater) the ice cover is complete.

\section{Inferences about climate of the period}

\subsection{Greenland Sea --- summers of 1810--1817}

The biases in Scoresby's air temperature measurements are not known exactly, but it is likely that his measurements are too warm rather than too cold. So for the years 1810--1815, where the reported temperatures are typically comparable to or colder than modern climatology, it is likely that the climate was somewhat colder than in 1979--2004. This is supported by the observation that he was operating at latitudes where sea-ice was rarely or never seen during 1979--2004, in spite of habitually working close to the ice edge.

In 1816 and 1817, the temperatures rose significantly, and the zone of operation moved north (presumably along with the sea-ice).

\subsection{Baffin Bay --- summer 1818}

At the beginning and end of this voyage, when the ship was crossing the relatively warm North Atlantic, the observed air temperatures were slightly above the modern average, and the observed SSTs slightly below (figure \ref{F3}). When the ship reached the modern-day sea-ice edge in the Labrador Sea the observed temperatures fell, both in absolute terms and relative to the modern values, and they stayed low throughout the circuit of Baffin Bay. The 1979--2004 sea ice coverage in the region is very variable, and it is likely that in 1818 was on the high side of the modern range, but not exceptionally so, except perhaps in June, when very cold conditions were experienced.

\subsection{Fram Strait --- summer 1818}

The {\it Dorothea} (figure \ref{F4}) saw falls in absolute and relative temperatures when it reached the modern sea-ice edge in Fram Strait. The modern sea-ice coverage at the latitudes the expedition reached is modest (none in warm years), and this is confirmed by Scoresby's observations in almost the same place. So, as in Baffin Bay the same year, the summer of 1818 was somewhat colder than in 1979--2004, with more sea-ice than usual.

\subsection{Lancaster Sound and Melville Island --- 1818--19}

The most immediately striking point in figure \ref{F5} is (as already noted by \citet{przybylak05Arctic}) the very cold winter temperatures. Even more remarkable, however, are the sea-ice conditions: the synchronous drop in SST in the observations and the climatology in summer 1819 show that the sea-ice edge in the Labrador sea was in much the same place in 1819 as in the modern day, but the late freezing of the sea ice around Melville Island is very unusual by modern standards, as is the early break-up of the ice the following year (note the anomalously warm period in the early summer of 1820). In all but the lowest sea-ice years in the 1979-2004 period the sea-ice around Melville Island does not break up at any time of year. 

\subsection{Foxe Basin --- 1821--3}

The 2.5-year record for the expedition to Foxe Basin (figure \ref{F2}) shows consistently cold summers (low air and sea temperature), but not the very cold winters seen on the 1819--20 expedition. The sea-ice conditions are also consistent with summers being significantly colder than the modern range. In both 1821 and 1822 the sea froze completely weeks before any year in the modern range, and in 1823 the thaw was similarly delayed.

\subsection{Greenland Sea --- summer of 1822}

On the other side of Greenland, in the summer of 1822 Scoresby was finding very different conditions in the Greenland Sea, where a low sea-ice year allowed him to venture unusually far north.

\subsection{Baffin Bay and Port Bowen --- 1824--5}

The last expedition (figure \ref{F6}) showed the same pattern as that of 1821--3: cold summers (in both air and sea temperatures) and early onset of completely frozen sea. Again winter temperatures are unexceptional.

\conclusions

The logs of whaling voyages and Royal Navy expeditions in the early nineteenth century offer a valuable source of information on the Arctic climate of the period. The observations are regularly and carefully made, and the combination of air-temperature, sea-surface-temperature, and ship movement information allow confident inferences to be made about climate change in spite of uncertainties about observing methods and instruments. 

Marine Arctic climate in 1810--25 (compared to 1979--2004) was marked by consistently cold summers, with a lot of sea-ice. But although the period was significantly colder than the modern average, there was a lot of variability. In the Greenland Sea, the summers following the Tambora eruption (1816 and 1817) were noticeably warmer and had lower sea-ice coverage than the years immediately preceding them, and the sea-ice in Lancaster Sound in 1819 and 1820 was low even by modern standards.

\begin{acknowledgement}
This work was funded by the Joint Information Systems Committee through the CORRAL project (Colonial Registers and Royal Navy Logbooks). PB and RA were also supported by the Joint DECC and Defra Integrated Climate Programme, DECC/Defra (GA01101), and RA also by the Queensland Climate Change Centre of Excellence (QCCCE). 
\end{acknowledgement}

\bibliographystyle{copernicus}
\bibliography{database,bibliography_extra}

\end{document}
